\begin{resumo}

A utilização de sistemas de software vem se tornando cada vez mais populares na sociedade moderna, acompanhando esse ritmo de crescimento
, surgem novos métodos e técnicas de aprendizagem que utilizam destes sistemas como uma forma de melhorar o ensino e aprendizagem. Este trabalho 
apresenta o passo a passo realizado na construção de uma ferramenta \textit{web} gamificada, cujo o objetivo é servir como um meio
de apoiar o ensino e aprendizagem de programação na disciplina "Algoritmos e Programação de Computadores", ofertada no campus Gama (FGA) da 
Universidade de Brasília (UnB). A ferramenta é baseada em estudos e ferramentas semelhantes, questionário, games mais jogados 
pelos estudantes e no \textit{framework octalysis} criado por Yukai Chou. Também é apresentado o conteúdo pegadagógico abrangido pela ferramenta
bem como o processo de escolha das técnicas de gamificação.

 \vspace{\onelineskip}
    
 \noindent
 \textbf{Palavras-chave}: Gamificação, Ferramenta, Requisitos, Desenvolvimento, Motivação, Aprendizagem, Programação, Arquitetura, Técnicas.
\end{resumo}
